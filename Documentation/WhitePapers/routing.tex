\documentclass{cmspaper}
\def\RCS$#1: #2 ${\expandafter\def\csname RCS#1\endcsname{#2}}
\RCS$Revision$
\RCS$Date$

\begin{document}
\begin{titlepage}
  \whitepaper
  \date{Revision \RCSRevision, \RCSDate}
  \title{CMS Data Handling: Routing System}

  \begin{Authlist}
    Tim~Barrass, Simon~Metson\Instfoot{bristol}{University of Bristol, Bristol, UK}
    Lassi~A.~Tuura\Instfoot{neu}{Northeastern University, Boston, USA}
  \end{Authlist}

  \begin{abstract}
    This white paper describes the components and algorithms used by
    the system we use to route files to their destinations.  The main
    algorith is the Routing Internet Protocol (RFC 2453) adapted for
    our purposes.  This documented is intended to evolve as we extend
    the implementation to handle more complex user requirements.
  \end{abstract} 

  \note{DRAFT version}
\end{titlepage}

\setcounter{page}{2}

\section{Overview}

[FIXME: Describe the objectives and what we are doing.]

\section{Forming the network: registration and simple routing}

This algorithm is an implementation of the Routing Internet Protocol
(RFC2453), one of the simplest (and most widely used) distance vector
based internet routing protocols. Our algorithm has certain
modifications---principally we have chosen not to have nodes
communicate with each other to determine current route
information---instead, updates are made asynchronously in a central
database.

For this version we should keep state information in the TMDB.
Instead of having distributed routing tables- with info for node a, b,
etc---we can have a global table that is partitioned on a field
indicating which node the information in a given row is for. We can
form a global routing table in the TMDB, with the fields [from : to :
gateway : hops : timestamp].

At deployment, initial entries must be made in the table to register
nodes with each other---e.g. if A and B are deployed then the entries

\begin{tabular}{lllll}
from	& to	& gate	& hops	& timestamp	\\ \hline
% ------------------------------------------------
A	& B	& B	& 1	& t0		\\
B	& A	& A	& 1	& t0
\end{tabular}

must be made.

From then on each agent manager periodically goes through a two or
three step series of interactions with the global routing table: all
agent managers should go through re-registration, and propagation of
information in order to build the routing table.  If failover is
required, then agent managers need to go through an expiry phase.

Re-registration is necessary as the nodes are not dynamically querying
each other.  The process of re-registration and propagation can be
incorporated into the same step, using the following algorithm

\begin{verbatim}
# determine who registered with me
@neighbours
  = (select gateway from t_routing where from = `me' and to = `gateway');

foreach $neighbour (@neighbours)
  # refresh our acquaintance with them
  update t_routing
    set timestamp = currenttime
    where from = `neighbour'
      and to = `me'
      and gateway = `me';

  # get routes in $routes{TO} = HOPS hash.
  foreach ($to, $hops) (select to, hops from t_routing where from = `neighbour')
     $routes{$to} = min ($hops, $routes{$to})

  # compare each route with my info and update
  foreach $destination (keys %routes)
    hops = routes{$destination} + 1;
    if (destination in @destinations)
      update t_routing
        set hops = $hops, timestamp = currenttime
	where from = `me'
	  and gateway = `neighbour'
	  and to = `destination';
    else
      insert into t_routing `all new values'
\end{verbatim}

If failover is not required, then the routing algorithm stops here.
However, if failover is required, then the algorithm needs to be
modified to handle nodes that disappear.  Disappearing nodes can be
handled by effectively setting the number of hops to that node to
infinity during an expiry phase.

During this phase the agent manager examines links with each of its
neighbours (e.g. links in it's own from-partition of the table for
which to == gateway) in the routing table.  If an entry has not been
refreshed within some critical period, the number of hops is
effectively set to infinity (the RIP algorithm suggest 16, assuming
that the protocols are useful only for networks where the longest
direct routes are less than 15 hops).

The information that a link is down is then gradually propagated
throughout the table as nodes ``ask'' their neighbours (or examine
their from-partitions) for their current estimate of hops for each
route.  When the node comes back up and re-registers, it will
automatically set the number of hops to 1.

[The algorithm could be trivially extended to include some measure of
dynamic network performance by entering ping round trip times instead
of hops into the routing table (note tho' that ping might not be
entirely suitable---pinging a tape stage disk server doesn't
necessarily give you an idea of fileserving performance...) Other
metrics such as the rate of the Mona Lisa file transfer rate metric
could also be used.]

\end{document}
